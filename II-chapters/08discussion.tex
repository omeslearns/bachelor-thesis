\chapter{Discussion} \label{ch:discussion}

Open and enterprise KGs are established research areas. Their construction, enrichment, schema and context are heavily based on consensus. For open KGs, a consensus is needed to make them interoperable with other KGs from the same domain, and for adhering to linked open data principles. We think it is no coincidence that the maintenance of open knowledge graphs is closely connected to other consensus-based efforts like Wikipedia. Enterprise KGs are proprietary and instead of being built from the established consensus in a domain, they define what the consensus for data representation in their application area is. This approach is closer to strict database schema. 

We believe that PKGs will take a different direction regarding consensus. The use case for consensus in PKGs will probably be about smaller communities that share consensus about a certain topic. For example, a hobbyist community could build a KG that individuals can clone, integrate into their PKG and personalize to their liking. We also regard PKGs as a promising approach in education. Rather than the linear and chronological information representations students are currently served, teachers might hand out knowledge graphs. These can be traversed based on interest and relevance to the individual student, which we think might result in more enthusiastic interaction with the material. Students could even be split into groups to work on small knowledge graphs that would be merged at the end of the course.

All of this hinges on everyday people having better access to personal knowledge management and structured knowledge. We think that pursuit of the ideas presented in this thesis might not only help people with their personal knowledge management, but also cause them to take more interest in opening up knowledge proliferation. A similar effect can be observed with software developers, who are required to work and think collaboratively regarding large codebases. They developed incredibly strong solutions for data reuse and composition which resulted in the most amazing open source projects. 