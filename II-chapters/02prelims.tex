\chapter{Preliminaries}
\section{Related Concepts and Technologies}

- Personal Information Management
- Memex / Tools for Thought
- Knowledge Bases
- Augmenting Human Intellect
- Data Privacy and Ownership
- Semantic Web Standards — RDF, OWL, SPARQL
- Ontologies and Open Vocabularies
- In-Memory Triple Store — RDF-JS spec
\section{Related Work}
% Put Related Work last, so you can connect what is missing here with the next chapter which is your approach (to fill in the gap that you explain at the end of the related work)

Related Work

Personal Information Management

Personal Knowledge Bases (merge with above)?

Semantic Web usability

**How would a memex2 based on RDF work?**

- In the cloud, with multi-device sync
- stored in RDF
- can integrate external  Data Sources
    - just the ones  the user is interested in
- structured information in the graph can be queried
- collaboration is possible
    - this needs access rights → handled by the server. based on RDF metadata?
        - how would you manage write tokens / keys in RDF? thats stupid actually
- 
- reference tools that mention the need of
    - privacy,
    - collaboration
        - read
        - comment
        - write
        - 
    - access control,

PIM Tools

- Notion, Roam, etc.

\section{unsorted}
do pkgs need to be spiderwebs, with the person in the center, or are graph datasets enough?

This is relevant when thinking about merging PKGs or collaboration on PKGs

Say you have a library and “Dieter” likes certain movies, while “Anna” likes other movies.
if there is no connection to the liked movies, how would the merged KG represent which person likes what?

- **Spider Web PKG?** While PKGs are by some Authors suggested to be Data “about” a Person and its Relationship to Entities (that are not necessarily public) in his Personal Domain, PKGs can also include a Persons knowledge about personal or public Entities, or just Knowledge Management of Information and Media Domains relevant to the user.
- Workshop on PKG Introduction
    
    The concept of **personal knowledge graphs (PKGs)** has been around for a while, in recognition of the need to represent structured information about entities that are personally related to a user. However, several open questions remain: (**comment**: I would argue the need is **information overload**, and pkgs are about information and knowledge RELEVANT to it’s user)
    
    - **Definition** — The notion of a personal knowledge graph has been established loosely, as a resource of structured information about entities personally related to its user. This definition needs crystallization: What is personal knowledge and how is it represented in a PKG? What differentiates a PKG from general KGs, how are they related? How can PKGs benefit from information stored in general KGs and how is the benefit realised? How is work on PKG related to work in areas such as commonsense KGs and entity/event-centric understanding?
    - **PKG construction/population** — What are the potential data sources (textual, visual, geolocation, etc.)? How to mitigate the ‘dual use’ of automated technology for PKG construction/population, since it can be used to extract and exploit personal knowledge about others?
    - **PKG utilities** — What novel application scenarios would PKGs enable and what role does/can novel techniques such as semantic technologies and knowledge modeling play in this respect? How do PKG compare to existing solutions to these applications?
    - **Practical realization** — Where would PKGs be stored and how would these interact with a range of external services, while considering access control as well as privacy concerns of users?
- knowledge graphs concern public / global entities
    - global entities / persons
    - global knowledge
    - problems
        - rules out many entities people interact with in private domain
        - not just information regarding yourself, but also non-public knowledge you have

While early imaginators of PKBs and PKGs were limited by technological hurdles, these hurdles have all but disappeared today.

- Semantic Web Standards
    - RDF
    - RDFS
    - RDF*
    - OWL
    - SPARQL

general KGs are about consensus schema, interlinking and inference. PKGs are about centralisation, PIM, collaboration……