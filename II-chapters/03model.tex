\chapter{Modeling PKGs with SWS}
\section{The Structural Framework}

- the fundamental data structure limits what you can store and express with the model — we want to express everything from atomic concepts (vocabulary) to large topics (Projects, Science)
- limitations of Trees/Documents
    - single parent Nodes
    - Freely associating Nodes is not possible
- benefits of Databases
    - association
    - Transclusion (embedding)
    - structure
    - Querying
- Benefits of Graph
    - adaptable schema
    - Query evaluation
- Benefits of SWS
    - semantics
    - open vocabularies
    - integrating external Graphs
- limitations of SWS / RDF
    - RDF*
    - performance
\section{Requirements of a PKG Graph Model}

- **Structured and Unstructured Knowledge.** Both academia (memex, augmenting human intellect) and industry (notion, roam) arrived at the conclusions that a personal knowledge base should be capable of representing structured information (databases, knowledge graphs), as well as unstructured Information (hypertext, markdwon) [cite memex 60 years later etc]
    - structured Knowledge — as close to SWS as possible
    - unstructured Knowledge — Markdown Notes
- **bi-directional links?** automatic generation of backlinks. (investigate owl property types; inverseProp etc.)
- **Node exist quantor.** Nodes need to be atomic entities. For RDF each Node has a Class Triple (fallback Class Resource)
- **Node IRI.** Every Node needs exactly one IRI
- **Relationsships?** In RDF Nodes with `rdfs:type rdfs:Property`
- **Arcs?** Called Triples in RDF (instances of Relationships)
- **Unique human readable label.** every Node needs a *unique?* human readable label (2 sources of uniqueness, IRI and Label)
    - unique, because for the user it’s the only visible “adress” of the entity?
    - IRIs are not human readable
    - pkgs are supposed to be operated by non-technical users.
\section{Types and Syntax}

- Types
    - Node (BlankNode?)
    - Relationship
    - Literal (String or Datatype)
- Casing
    - Nodes are `PascalCase`
    - Relationships are `camelCase`
    - Classes are SCREAMING SNAKE

\section{The different Data Layers}

1. static external Schema (Ontologies)
2. adaptive personal Schema
3. Structured Data and Notes
    - extra Note Data Layer?
4. temporary Data Layer (inferred from Notes, external Data, etc)
5. Metadata for display properties (order, sorting, filters, alignment)
\section{CRUD operations and effects}
\section{Advanced Semantics with OWL}
\section{unsorted}
the data storage aspect is the most fundamental and limiting factor when designing a PKB Tool, because it defines what you can express and store.

Text, Trees and Graphs

trees are obv. a subset of graphs, however thats like saying a masterfully crafted sculpture is a subset of clay. specification creates unique structural properties, and a graph lacks the hierarchy and probably more importantly the chronology that a Book / Tree provides. This chronology provides the user with a sense of location and navigation, that is urgently missing from graphs.

Data model, structural framework, architecture

examples of standardised RDF formats; iCal, Contacts, Musicbrianz, etc

PKM putting everything into hierarchy makes free association impossible. This leads to copying of Information, with some of it outdated.

anyone who has seen a TOC knows text is a ordered tree. It’s time we went from “Text” Information Media like Articles and Books, to Graph based Media, like Databases and Knowledge Graphs

how to model PKG as a data resource for Apps

- schema needed per data domain
    - media
    - health
    - contacts
    - calendar
    - resources
    - …

Thesis: RDF / graph naturally lends itself to PKG TFT functionality, because it actually is a graph. So things like bi-links etc get easier to implement

- Notes
    
    converting unstructured → structured knowledge must be frictionless
    
    - It is worth mentioning that
        1.  lots of people are not satisfied with SWS and even the creators acknowledge Problems in design (better rdf repo) and 
        2.  standards can change and be extended (RDF*) thus the basis for the model is graph theory, not SWS theory. For now the selected graph model will be RDF