\chapter{Preliminaries} \label{ch:preliminaries}
This chapter provides definitions for concepts used throughout the thesis. For descriptions of terms and phrases please refer to the glossary.

\begin{itemize}
    \item Knowledge base
    \item knowledg graph 
    \item RDF 
    \item TURTLE
\end{itemize}

\begin{definition}[Graph]
    A \textbf{graph} is a tuple $G = (V, E)$, where $V$ is a set of \textbf{vertices} and $\displaystyle E\subseteq \{\{x,y\}\mid x,y\in V\;{\textrm {and}}\;x\neq y\}$ is a set of \textbf{edges}.
\end{definition}

\begin{definition}[Directed graph]
    A \textbf{directed graph} is a tuple $G = (V,E)$, where $V$ is a set of vertices and $\displaystyle E\subseteq \left\{(x,y)\mid (x,y)\in V^{2}\;{\textrm {and}}\;x\neq y\right\}$ is a set of \textbf{directed edges}.
\end{definition}

\begin{definition}[Labeled graphs]
    An \textbf{edge-labeled graph} is a graph that has a labelling function $l_E : E \rightarrow X$, where $X$ is a set of edge labels.
    A \textbf{vertex-labeled graph} is a graph that has a labelling function $l_V : V \rightarrow Y$, where $Y$ is a set of vertex labels.
    A \textbf{labeled graph} is a graph that is both edge-labeled and vertex-labeled.
\end{definition}

\begin{definition}[Multigraph]
    A \textbf{multigraph} is a tuple $\displaystyle G=(V,E,\phi )$, where $V$ is a set of vertices, $E$ is a set of edges and $\displaystyle \phi :E\to \{\{x,y\}\mid x,y\in V\;{\textrm {and}}\;x\neq y\}$ is an incidence function mapping every edge to an unordered pair of vertices.
\end{definition}

\begin{remark}
    These definitions can be combined to get a directed labeled multigraph.
\end{remark}


% \includegraphics[width=0.7\textwidth]{defKG.png}
% \begin{definition}[Knowledge Graph]
%     A \textbf{Knowledge Graph} is a directed labeled multigraph.
% \end{definition}

\begin{remark}
    The definition of a KG remains contentious \cite{Hogan2021KG,commonsenseKG}.
\end{remark}

\begin{figure}[H]
    \centering
    \includegraphics[width=0.5\textwidth]{kg}
    \caption[]{A Knowledge Graph}
\end{figure}

\begin{figure}[H]
    \centering
    \includegraphics[width=0.75\textwidth]{rdf}
    \caption[]{An RDF Graph}
\end{figure}

% For more background on these concepts, we refer the reader to the primer on the subject by Hogan et al \cite{Hogan2021KG} and the official W3C specifications \cite{rdf, rdfs, owl, turtle, sparql, xsd}.

% comment: This is not an RDF graph. 

% This picture is not correct in several ways: 

% 1. In an RDF triple, properties are labels of the edges, not nodes themselves. So, in the RDF triple (Q9439, P22, Q157009) P22 is just the label of the edge, not a node. This also solves the problem that some of your edges do not have labels, which is not allowed in RDF. 

% 2. The labels for properties like label_en and label_de is not the standard way of representing human-readable labels in RDF. For that we use rdfs:label, and it is ONE property, without distinction of the the language in the property. If you want to say that the value is in a specific language, then we use language-tagged literals … for example “Queen Victoria”@en. 

% 3. It is a convention when plotting RDF graphs to use rectangles for depicting  literals, and ovals for IRIS 