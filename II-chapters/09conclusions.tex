\chapter{Conclusions} \label{ch:conclusions}
% This chapter is very generic, it needs to be more specific. What did we accomplish with this thesis? What is left to do? How this can be integrated with SOLID? 

% all the proposals in this paper are mainly things a single person came up with, guided by ideas from research or inventions. This work needs to be standardized by large bodies of knowledge and communities

- Problem with PKGs: Schema, Context, Identity
- The Missing Global Ontology standard for Personal Knowledge
    - How tightly should PKGs integrate concensus based open ontologies?
- people have been struggling to develop better mediums than text, for mainly x reasons:
    - people learn to read and write in school, they don’t learn information science or databases

- Semantic Web also promises global consistency of terminology and unambiguous identifiers for shared concepts. This latter capability is what makes Semantic Web technologies an enticing component to consider for a PKB system. [memex60]

How will the tool be used by real users?

Why Graphs for personal Knowledge Bases?

comparison of Text Documents / PDFs / Books to the Graph structure (hierarchy, chronology, sense of orientation / navigation)

- Use Cases of KG
    
    - Open
    
    - Enterprise
    
    - Personal
    
    - Collaborative PKGs?


For the Prototype, I would have liked to be able to do user tests and integrate with a backend like described in [pkg ecosystem]

% ------------------------------------------------------------------------------
% model is great in comparison to what?

This thesis has explored the PKG Ecosystem that would develop from PKGs serving as the foundation for lifelong Personal Knowledge Management.

We have shown that Information and Knowledge from hierarchical Documents and Databases can be mapped to a directed labeled Graph Model. Implemented with Semantic Web Standards, we can levererage Semantic Web Technologies for PKGs and PKM. The Model provides great flexibility, expressivity and interoperability.

Semantic Markdown was proposed as a Syntax extension for Markdown, to mix RDF Relationships in plain text.

PKGs are still a mostly uncharted research area with basic research questions regarding definition, generation and enrichment. 

For PKGs to serve as the foundation for PKM, we need standards for Data formats for Media, Health, Contacts, etc. and for Semantic Markdown or similar Hypertext.

We also need to consider the Backends that manage security, query evaluation and access rights. 

The SOLID Project pioneered by Tim Berners Lee already supports RDF and SPARQL and might be a promising approach for PKGs.



with more time, I would have liked to 

develop the prototype more in depth

do usability tests

have a more detailed look at some of the requirements, like version control and access rights for RDF graphs