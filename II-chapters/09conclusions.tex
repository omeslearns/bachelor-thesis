\chapter{Conclusions} \label{ch:conclusions}
% This chapter is very generic, it needs to be more specific. 
% What did we accomplish with this thesis? 
% What is left to do? 
% How this can be integrated with SOLID? 

In this thesis, we have investigated PKGs and the PKG ecosystem as approaches to personal knowledge management and the information overload problem. 
We have outlined the functionalities required of the PKG ecosystem and established that PKM tools based on proprietary formats or text files are not viable candidates for the PKG ecosystem. We have proposed RDF as the data model for PKGs and outlined how to store both structured and unstructured data with it. To achieve this we proposed the development of a markdown extension that can express RDF statements. As a proof of concept, we implemented the prototype of a web application that can store structured and unstructured knowledge as an RDF graph.

given more time, we would have developed the prototype with more of a focus on usability by including user tests.

Version control and access rights for PKGs are also topics that we would have liked to investigate in more detail.

Limitations in our approach are that we could not create standards for the RDF data model or semantic markdown. This would require more resources and consensus.

The SOLID project is an initiative by Tim Berners Lee to provide individuals with online data stores, so-called "Pods". These pods could potentially serve as the backend we envisioned for the PKG ecosystem, because they use the RDF data model, and are compatible with several graph query languages \cite{solid}. 
