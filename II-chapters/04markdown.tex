\chapter{Semantic Markdown with RDF}
- differentiate semantic nodes vs text nodes
    - semantic nodes are entities with types
    - text nodes are unstructured knowledge that exists as nodes in the graph, but are also arranged in hierarchy attached on a semantic entity
\begin{table}[h]
    \centering
    \begin{tabular}{ |c|c|c| }
        \hline
        cell1 & cell2 & cell3 \\ 
        \hline
        cell4 & cell5 & cell6 \\  
        \hline
        cell7 & cell8 & cell9 \\
        \hline
    \end{tabular}
    \caption{a table}
\end{table}
\section{Markdown Outliner in RDF}

It is possible to display Text stored in a Graph like it was plain text or markdown.

There is however the question of how to handle metadata about display order and style.

% :IRI rdf:type :Person, :Artist, :Musician
% :IRI rdfs:label "Lady Gaga"
% :IRI foaf:givenName "Stefani Joanne Angelina"
% :IRI foaf:familyName "Germanotta"
% :IRI :note :note_000
% :note_000 rdf:type :Note
% :note_000 rdfs:label "Lady Gaga got a lot of attention for wearing unorthodox dresses"
% :note_000 :note :note_001
% :note_001 rdf:type :Note
% :note_001 rdfs:label "example for exotic dress nr. 1"
% :note_000 :note :note_002
% :note_002 rdf:type :Note
% :note_002 rdfs:label "example for exotic dress nr. 2"
% :note_000 :note :note_003
% :note_003 rdf:type :Note
% :note_003 rdfs:label "example for exotic dress nr. 3"
results in: 

**Lady Gaga**

rdfs:type→  [Person](https://www.notion.so/Person-dece8f28272f428789d1a0cd875982d4) , [Artist](https://www.notion.so/Artist-62b70a2518e447528a7165ed48022d22) , [Musician](https://www.notion.so/Musician-cd6095c9ba374616ad2ca44e06981729) 

foaf:givenName→ “Stefani Joanne Angelina”

foaf:familyName→ "Germanotta"

[Lady Gaga got a lot of attention for wearing unorthodox dresses during events and performances](https://www.notion.so/Lady-Gaga-got-a-lot-of-attention-for-wearing-unorthodox-dresses-during-events-and-performances-9bd88b658e5a44fd8054b8e670cd5ec3) 

[example for exotic dress nr. 1](https://www.notion.so/example-for-exotic-dress-nr-1-b4b1125f9afc4c9e86c5a49e1a12d95c) 

[example for exotic dress nr. 2](https://www.notion.so/example-for-exotic-dress-nr-2-a9a21d52e922480d8c40d68cd9770cd4) 

[example for exotic dress nr. 3](https://www.notion.so/example-for-exotic-dress-nr-3-02c3074cac824313b1fdb957f7c9ca38) 

Note that all of the unstructured text notes are still their own entities, just designated with a rdf:type of <:Note>
\section{Advanced Markdown Flavors}
Markdown although not standardised, is embraced by the Web Community and continuously extended. There are several Flavors (Github, CommonMark, etc.). Approximately the following expressivity levels of Markdown have developed:

- **Basic Markdown.** Includes mostly Text formatting:
    - Headings, **Bold**, *Italic*, ~~Strikethrough~~, Quotes, `Code`
    - Lists
    - Images and Links
- **Extended Markdown.** Includes advanced Formatting options:
    - Tables
    - Heading ID’s (in-document navigation)
    - Syntax highlighted Code Block
    - Footnotes
    - Todos
    - Emoji
    - Highlighting
    - Sub- and Superscript
    - Table of Content
    - Callouts
    - Comments
    - Captions
- **“Hypertext” Markdown.** Recently Note-Taking Tools have adopted shared extended Syntax, even enabling toolwide Hyperlinks:
    - `[[` automatically gets converted to links
    - `((` mention
    - `{{` embeds
    - `\$\$` LaTeX code
    - `^^` Highlight Text
\section{A Semantic Markdown extension}
This takes the approach of Hyptertext markdown and adds semantic relationships to it that can be used inside the PKG. These semantic relationships are then inferred into one of the PKG Datasets graph layers.

- inferred links from markdown nodes:
    - mentions
    - embeds
    - links
- A proposal for semantic markdown could look like this:
    - `[[` link ~ rdfs:seeAlso
    - `((` mention
    - `{{` embed
    - `>>` Relationship
    - `<<` inverse Relationship
    - `::`
    - `@@`
    - `\&\&`
    - `;;`
    - `%%`
\section{Advanced Semantics (Owl etc)}
These need some kind of Syntax…
- inverse
    - Felix —parent>> —child<< Omes
- functional : x eindeutig auf y
    - company —foundingYear== 1960
- inverseFunctional : y eindeutig auf x
    - Marion ==bioMotherOf— Omes
- transitive
    - ancestor
- symmetric
    - obama :hasspouse Michelle
- antisymmetric
    - :parent
- reflexive
    - knows
- irreflexive
    - married