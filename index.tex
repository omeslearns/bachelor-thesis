%Packages are included below, these are akin to libraries in other programming languages. As far as I can tell, there is no reason to reduce the number of packages included in a document. It might compile faster, but overleaf compiles fast so this should not be an issue.
\documentclass[
fontsize=11pt,
paper=a4,
abstract=true,
headings=standardclasses,
chapterprefix=false,
numbers=noenddot,
parskip=half+, % comment this out if you do not want an empty half line between paragraphs, but please read the KomaScript Guide and search for parskip (around page 82): ftp://ftp.dante.de/pub/tex/macros/latex/contrib/koma-script/scrguide.pdf
]{scrreprt}
% \documentclass[11pt]{report} %tells the compiler that this is a 'report' style document, and the main font size.
\usepackage{setspace} %allows the use of '\doublespace' to set line spacing
\usepackage[utf8]{inputenc} %inclusion of this is optional, overleaf includes it in its compiler so it is not necessary, it may be necessary for other compilers.
\usepackage{wrapfig} %if it is desirable to wrap text (see https://www.overleaf.com/learn/latex/Wrapping_text_around_figures).
\usepackage{graphicx} %this allows graphics to be put in easily
\usepackage{float}%this allows you to put them in good places
\usepackage[version=4]{mhchem} %this is good for chemical reactions
\usepackage{amsmath} %maths package
\usepackage{amssymb} %symbol package
\usepackage{textcomp,gensymb} %more symbols (eg \degree)
\usepackage{appendix} %self explanatory
\usepackage{colortbl} %good for colouring in cells on a table
\usepackage{rotating} %allows you to rotate graphics
\usepackage{bm} %helps bold things
\usepackage{multirow} %for tables
\usepackage{longtable} %for long tables
\usepackage{booktabs} %more tables
\usepackage{caption} %allows captions for graphics
\usepackage[nottoc]{tocbibind} %adds the bibliography to the table of contents
\usepackage{subcaption} %allows subcaption for multiple images in one graphic

\PassOptionsToPackage{hyphens}{url}\usepackage{hyperref}
\usepackage{hyperref} %this is great for putting hyperreferences in the document.
\usepackage[table]{xcolor} %more colouring of tables
\definecolor{Gray}{gray}{0.9} %this defines a colour to be used and gives it a name. This is a colour called 'Gray' it is 'gray' with transparency 0.9
% \input{preamble/arduinoLanguage.tex}  %I used this to format arduino code well in my report. I'm sure there are other packages/methods for other languages.
\hypersetup{
    bookmarks=true,         % show bookmarks bar?
    unicode=false,          % non-Latin characters in Acrobat’s bookmarks
    pdftoolbar=true,        % show Acrobat’s toolbar?
    pdfmenubar=true,        % show Acrobat’s menu?
    pdffitwindow=false,     % window fit to page when opened
    pdfstartview={FitH},    % fits the width of the page to the window
    pdftitle={My title},    % title
    pdfauthor={Author},     % author
    pdfsubject={Subject},   % subject of the document
    pdfcreator={Creator},   % creator of the document
    pdfproducer={Producer}, % producer of the document
    pdfkeywords={keyword1, key2, key3}, % list of keywords
    pdfnewwindow=true,      % links in new PDF window
    linktoc=all,
    colorlinks=true,       % false: boxed links; true: colored links
    linkcolor=black,          % color of internal links (change box color with linkbordercolor)
    citecolor=green,        % color of links to bibliography
    filecolor=cyan,         % color of file links
    urlcolor=magenta        % color of external links
}

%%%%%%%%%%%%%%%%%%%%%%%%%%%%%%%%%%%%%%%%%%%
% Preamble
%%%%%%%%%%%%%%%%%%%%%%%%%%%%%%%%%%%%%%%%%%%
\begin{document} % you have to do this to start the document
\pagenumbering{roman}
\begin{titlepage}
    \begin{center}
        % Lehrstuhl Mathematics and Informatics\\
        % this title page can be played around with to better serve your project. the necessary parts to include are the TITLE at the top, the BLURB about the thesis presented etc. Name, faculty member, date, type of engineering...
        \vspace*{1cm}
        \Huge
        \textbf{Standards based Personal\\ Knowledge Graphs}\\
        
        \vspace*{1cm}
        
        \Large
        Omes Felix Baltes
      
        \vfill
        
        \large
        Bachelor's Thesis -- 15. July, 2022\\
        
        Ruhr University Bochum\\
        Faculty of Mathematics\\
        
        \vspace{1cm}
        
        
        Supervisor: Prof. Maribel Acosta\\
        Advisor: Prof. Ajsa Fischer\\
 
    \end{center}
\end{titlepage}
\vspace*{\fill}

\begin{center}
© 2022\\
Omes Baltes\\
ALL RIGHTS RESERVED
\end{center}
\thispagestyle{empty}
\begin{abstract}
\thispagestyle{plain}
Knowledge graphs have been leveraged in enterprise and open data  for more than a decade, but for personal use, they are just emerging. Most of the tools for personal knowledge graphs are based on proprietary tech and data formats. This Paper explores how data-centric PKGs can be modeled using Semantic Web Standards and how to create usability focused Interfaces for them.
\end{abstract} 
% \begin{center}
    
\vspace*{4cm}
My gratitude goes to 

% Maribel for the space to grow \\
% Sebastian for the paths to wander \\
% Hendrik \& Lara for the company on the way
\end{center}

\tableofcontents 
\listoffigures 
\listoftables 



%%%%%%%%%%%%%%%%%%%%%%%%%%%%%%%%%%%%%%%%%%%
% Chapters
%%%%%%%%%%%%%%%%%%%%%%%%%%%%%%%%%%%%%%%%%%%
\chapter{Introduction}
Here you should put a short introduction to your chapter. What is covered? In how much detail? Imagine you were coming back to this in 10 years time and wanted to find that one key equation, this part of the chapter should orient the reader to help find that information.

\section{Overview} \label{sec:overview}

\section{Vision}

\section{Contributions}

\section{Outline}

\pagenumbering{arabic}
\chapter{Preliminaries}
\section{Related Concepts and Technologies}

- Personal Information Management
- Memex / Tools for Thought
- Knowledge Bases
- Augmenting Human Intellect
- Data Privacy and Ownership
- Semantic Web Standards — RDF, OWL, SPARQL
- Ontologies and Open Vocabularies
- In-Memory Triple Store — RDF-JS spec
\section{Related Work}
% Put Related Work last, so you can connect what is missing here with the next chapter which is your approach (to fill in the gap that you explain at the end of the related work)

Related Work

Personal Information Management

Personal Knowledge Bases (merge with above)?

Semantic Web usability

**How would a memex2 based on RDF work?**

- In the cloud, with multi-device sync
- stored in RDF
- can integrate external  Data Sources
    - just the ones  the user is interested in
- structured information in the graph can be queried
- collaboration is possible
    - this needs access rights → handled by the server. based on RDF metadata?
        - how would you manage write tokens / keys in RDF? thats stupid actually
- 
- reference tools that mention the need of
    - privacy,
    - collaboration
        - read
        - comment
        - write
        - 
    - access control,

PIM Tools

- Notion, Roam, etc.

\section{unsorted}
do pkgs need to be spiderwebs, with the person in the center, or are graph datasets enough?

This is relevant when thinking about merging PKGs or collaboration on PKGs

Say you have a library and “Dieter” likes certain movies, while “Anna” likes other movies.
if there is no connection to the liked movies, how would the merged KG represent which person likes what?

- **Spider Web PKG?** While PKGs are by some Authors suggested to be Data “about” a Person and its Relationship to Entities (that are not necessarily public) in his Personal Domain, PKGs can also include a Persons knowledge about personal or public Entities, or just Knowledge Management of Information and Media Domains relevant to the user.
- Workshop on PKG Introduction
    
    The concept of **personal knowledge graphs (PKGs)** has been around for a while, in recognition of the need to represent structured information about entities that are personally related to a user. However, several open questions remain: (**comment**: I would argue the need is **information overload**, and pkgs are about information and knowledge RELEVANT to it’s user)
    
    - **Definition** — The notion of a personal knowledge graph has been established loosely, as a resource of structured information about entities personally related to its user. This definition needs crystallization: What is personal knowledge and how is it represented in a PKG? What differentiates a PKG from general KGs, how are they related? How can PKGs benefit from information stored in general KGs and how is the benefit realised? How is work on PKG related to work in areas such as commonsense KGs and entity/event-centric understanding?
    - **PKG construction/population** — What are the potential data sources (textual, visual, geolocation, etc.)? How to mitigate the ‘dual use’ of automated technology for PKG construction/population, since it can be used to extract and exploit personal knowledge about others?
    - **PKG utilities** — What novel application scenarios would PKGs enable and what role does/can novel techniques such as semantic technologies and knowledge modeling play in this respect? How do PKG compare to existing solutions to these applications?
    - **Practical realization** — Where would PKGs be stored and how would these interact with a range of external services, while considering access control as well as privacy concerns of users?
- knowledge graphs concern public / global entities
    - global entities / persons
    - global knowledge
    - problems
        - rules out many entities people interact with in private domain
        - not just information regarding yourself, but also non-public knowledge you have

While early imaginators of PKBs and PKGs were limited by technological hurdles, these hurdles have all but disappeared today.

- Semantic Web Standards
    - RDF
    - RDFS
    - RDF*
    - OWL
    - SPARQL

general KGs are about consensus schema, interlinking and inference. PKGs are about centralisation, PIM, collaboration……
\chapter{Modeling PKGs with SWS}
\section{The Structural Framework}

- the fundamental data structure limits what you can store and express with the model — we want to express everything from atomic concepts (vocabulary) to large topics (Projects, Science)
- limitations of Trees/Documents
    - single parent Nodes
    - Freely associating Nodes is not possible
- benefits of Databases
    - association
    - Transclusion (embedding)
    - structure
    - Querying
- Benefits of Graph
    - adaptable schema
    - Query evaluation
- Benefits of SWS
    - semantics
    - open vocabularies
    - integrating external Graphs
- limitations of SWS / RDF
    - RDF*
    - performance
\section{Requirements of a PKG Graph Model}

- **Structured and Unstructured Knowledge.** Both academia (memex, augmenting human intellect) and industry (notion, roam) arrived at the conclusions that a personal knowledge base should be capable of representing structured information (databases, knowledge graphs), as well as unstructured Information (hypertext, markdwon) [cite memex 60 years later etc]
    - structured Knowledge — as close to SWS as possible
    - unstructured Knowledge — Markdown Notes
- **bi-directional links?** automatic generation of backlinks. (investigate owl property types; inverseProp etc.)
- **Node exist quantor.** Nodes need to be atomic entities. For RDF each Node has a Class Triple (fallback Class Resource)
- **Node IRI.** Every Node needs exactly one IRI
- **Relationsships?** In RDF Nodes with `rdfs:type rdfs:Property`
- **Arcs?** Called Triples in RDF (instances of Relationships)
- **Unique human readable label.** every Node needs a *unique?* human readable label (2 sources of uniqueness, IRI and Label)
    - unique, because for the user it’s the only visible “adress” of the entity?
    - IRIs are not human readable
    - pkgs are supposed to be operated by non-technical users.
\section{Types and Syntax}

- Types
    - Node (BlankNode?)
    - Relationship
    - Literal (String or Datatype)
- Casing
    - Nodes are `PascalCase`
    - Relationships are `camelCase`
    - Classes are SCREAMING SNAKE

\section{The different Data Layers}

1. static external Schema (Ontologies)
2. adaptive personal Schema
3. Structured Data and Notes
    - extra Note Data Layer?
4. temporary Data Layer (inferred from Notes, external Data, etc)
5. Metadata for display properties (order, sorting, filters, alignment)
\section{CRUD operations and effects}
\section{Advanced Semantics with OWL}
\section{unsorted}
the data storage aspect is the most fundamental and limiting factor when designing a PKB Tool, because it defines what you can express and store.

Text, Trees and Graphs

trees are obv. a subset of graphs, however thats like saying a masterfully crafted sculpture is a subset of clay. specification creates unique structural properties, and a graph lacks the hierarchy and probably more importantly the chronology that a Book / Tree provides. This chronology provides the user with a sense of location and navigation, that is urgently missing from graphs.

Data model, structural framework, architecture

examples of standardised RDF formats; iCal, Contacts, Musicbrianz, etc

PKM putting everything into hierarchy makes free association impossible. This leads to copying of Information, with some of it outdated.

anyone who has seen a TOC knows text is a ordered tree. It’s time we went from “Text” Information Media like Articles and Books, to Graph based Media, like Databases and Knowledge Graphs

how to model PKG as a data resource for Apps

- schema needed per data domain
    - media
    - health
    - contacts
    - calendar
    - resources
    - …

Thesis: RDF / graph naturally lends itself to PKG TFT functionality, because it actually is a graph. So things like bi-links etc get easier to implement

- Notes
    
    converting unstructured → structured knowledge must be frictionless
    
    - It is worth mentioning that
        1.  lots of people are not satisfied with SWS and even the creators acknowledge Problems in design (better rdf repo) and 
        2.  standards can change and be extended (RDF*) thus the basis for the model is graph theory, not SWS theory. For now the selected graph model will be RDF
\chapter{Semantic Markdown with RDF}
- differentiate semantic nodes vs text nodes
    - semantic nodes are entities with types
    - text nodes are unstructured knowledge that exists as nodes in the graph, but are also arranged in hierarchy attached on a semantic entity
\begin{table}[h]
    \centering
    \begin{tabular}{ |c|c|c| }
        \hline
        cell1 & cell2 & cell3 \\ 
        \hline
        cell4 & cell5 & cell6 \\  
        \hline
        cell7 & cell8 & cell9 \\
        \hline
    \end{tabular}
    \caption{a table}
\end{table}
\section{Markdown Outliner in RDF}

It is possible to display Text stored in a Graph like it was plain text or markdown.

There is however the question of how to handle metadata about display order and style.

% :IRI rdf:type :Person, :Artist, :Musician
% :IRI rdfs:label "Lady Gaga"
% :IRI foaf:givenName "Stefani Joanne Angelina"
% :IRI foaf:familyName "Germanotta"
% :IRI :note :note_000
% :note_000 rdf:type :Note
% :note_000 rdfs:label "Lady Gaga got a lot of attention for wearing unorthodox dresses"
% :note_000 :note :note_001
% :note_001 rdf:type :Note
% :note_001 rdfs:label "example for exotic dress nr. 1"
% :note_000 :note :note_002
% :note_002 rdf:type :Note
% :note_002 rdfs:label "example for exotic dress nr. 2"
% :note_000 :note :note_003
% :note_003 rdf:type :Note
% :note_003 rdfs:label "example for exotic dress nr. 3"
results in: 

**Lady Gaga**

rdfs:type→  [Person](https://www.notion.so/Person-dece8f28272f428789d1a0cd875982d4) , [Artist](https://www.notion.so/Artist-62b70a2518e447528a7165ed48022d22) , [Musician](https://www.notion.so/Musician-cd6095c9ba374616ad2ca44e06981729) 

foaf:givenName→ “Stefani Joanne Angelina”

foaf:familyName→ "Germanotta"

[Lady Gaga got a lot of attention for wearing unorthodox dresses during events and performances](https://www.notion.so/Lady-Gaga-got-a-lot-of-attention-for-wearing-unorthodox-dresses-during-events-and-performances-9bd88b658e5a44fd8054b8e670cd5ec3) 

[example for exotic dress nr. 1](https://www.notion.so/example-for-exotic-dress-nr-1-b4b1125f9afc4c9e86c5a49e1a12d95c) 

[example for exotic dress nr. 2](https://www.notion.so/example-for-exotic-dress-nr-2-a9a21d52e922480d8c40d68cd9770cd4) 

[example for exotic dress nr. 3](https://www.notion.so/example-for-exotic-dress-nr-3-02c3074cac824313b1fdb957f7c9ca38) 

Note that all of the unstructured text notes are still their own entities, just designated with a rdf:type of <:Note>
\section{Advanced Markdown Flavors}
Markdown although not standardised, is embraced by the Web Community and continuously extended. There are several Flavors (Github, CommonMark, etc.). Approximately the following expressivity levels of Markdown have developed:

- **Basic Markdown.** Includes mostly Text formatting:
    - Headings, **Bold**, *Italic*, ~~Strikethrough~~, Quotes, `Code`
    - Lists
    - Images and Links
- **Extended Markdown.** Includes advanced Formatting options:
    - Tables
    - Heading ID’s (in-document navigation)
    - Syntax highlighted Code Block
    - Footnotes
    - Todos
    - Emoji
    - Highlighting
    - Sub- and Superscript
    - Table of Content
    - Callouts
    - Comments
    - Captions
- **“Hypertext” Markdown.** Recently Note-Taking Tools have adopted shared extended Syntax, even enabling toolwide Hyperlinks:
    - `[[` automatically gets converted to links
    - `((` mention
    - `{{` embeds
    - `\$\$` LaTeX code
    - `^^` Highlight Text
\section{A Semantic Markdown extension}
This takes the approach of Hyptertext markdown and adds semantic relationships to it that can be used inside the PKG. These semantic relationships are then inferred into one of the PKG Datasets graph layers.

- inferred links from markdown nodes:
    - mentions
    - embeds
    - links
- A proposal for semantic markdown could look like this:
    - `[[` link ~ rdfs:seeAlso
    - `((` mention
    - `{{` embed
    - `>>` Relationship
    - `<<` inverse Relationship
    - `::`
    - `@@`
    - `\&\&`
    - `;;`
    - `%%`
\section{Advanced Semantics (Owl etc)}
These need some kind of Syntax…
- inverse
    - Felix —parent>> —child<< Omes
- functional : x eindeutig auf y
    - company —foundingYear== 1960
- inverseFunctional : y eindeutig auf x
    - Marion ==bioMotherOf— Omes
- transitive
    - ancestor
- symmetric
    - obama :hasspouse Michelle
- antisymmetric
    - :parent
- reflexive
    - knows
- irreflexive
    - married
\chapter{PKG Tools}

\section{Requirements for a PKG App}
\section{Analysis of the Software Landscape}
\section{Applying the Model}
\section{Abstracting away Technical Details of Semantic Web Technologies}
\section{Testing Usability}
\chapter{Discussion}

- people have been struggling to develop better mediums than text, for mainly x reasons:
    - people learn to read and write in school, they don’t learn information science or databases
    - 

\section{How will the tool be used by real users?}

\section{The Missing Global Ontoloty standard for Personal Knowledge}

\section{Why Graphs for personal Knowledge Bases?}

comparison of Text Documents / PDFs / Books to the Graph structure (hierarchy, chronology, sense of orientation / navigation)

- Use Cases of KG
    - Open
    - Enterprise
    - Personal
    - Collaborative PKGs?
\chapter{Conclusions}

\section{SOLID Collaboration}

\section{Research Questions}

- How can PKGs be stored in the cloud
- Access rights
- Knowledge graph summarization

%%%%%%%%%%%%%%%%%%%%%%%%%%%%%%%%%%%%%%%%%%%
% Appendices
%%%%%%%%%%%%%%%%%%%%%%%%%%%%%%%%%%%%%%%%%%%
% \appendix
% \input{appendices/equations.tex}
% \input{appendices/sessionprotocol.tex}
% \input{appendices/code.tex}

%%%%%%%%%%%%%%%%%%%%%%%%%%%%%%%%%%%%%%%%%%%
% Bibliography
%%%%%%%%%%%%%%%%%%%%%%%%%%%%%%%%%%%%%%%%%%%
\bibliographystyle{ieeetr}
\bibliography{references}

\end{document}